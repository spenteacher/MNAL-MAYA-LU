\documentclass[10pt]{beamer}
\usepackage[utf8]{inputenc}
\usepackage[spanish]{babel}
\usepackage{amsmath, amssymb, amsthm}
\usepackage{hyperref}
\usepackage{graphicx}

\usetheme{Madrid}
\usecolortheme{default}

\title{Descomposición LU en Simulación Computacional}
\subtitle{Fundamentos Teóricos y Aplicación a Sistemas de Partículas}
\author{S. Zaragoza Pastor \and J. Hernández Barrero \and B. Liu \and A. Fernández García \and P. Botella Guillén}
\date{}

\begin{document}

\begin{frame}
  \titlepage
\end{frame}

\begin{frame}{Contenido}
  \tableofcontents
\end{frame}

% ============================================================
% SECCIÓN 1: INTRODUCCIÓN
% ============================================================
\section{Introducción}

\begin{frame}{Motivación}
La resolución de sistemas lineales $Ax = b$ es fundamental en álgebra lineal computacional.

\vspace{0.3cm}
\textbf{Pregunta clave:} ¿Podemos aprovechar cálculos previos al resolver múltiples sistemas con la misma matriz $A$ pero diferentes vectores $b$?

\vspace{0.3cm}
\textbf{Respuesta:} La descomposición LU separa el problema en dos fases:
\begin{enumerate}
  \item \textbf{Factorización:} $A = LU$ (una sola vez)
  \item \textbf{Resolución:} Resolver $Ly = b$ y $Ux = y$ (para cada $b$)
\end{enumerate}
\end{frame}

\begin{frame}{Aplicación Práctica}
\textbf{Sistema de partículas conectadas por resortes:}
\begin{itemize}
  \item Modelado mediante ecuaciones de equilibrio estático
  \item Resulta en sistemas lineales grandes $Kx = f$
  \item La matriz $K$ (rigidez) es simétrica y definida positiva
  \item Necesidad de resolver múltiples configuraciones eficientemente
\end{itemize}
\end{frame}

% ============================================================
% SECCIÓN 2: FUNDAMENTOS TEÓRICOS
% ============================================================
\section{Fundamentos Teóricos de la Descomposición LU}

\begin{frame}{Matriz Triangular Inferior}
\textbf{Definición:}
Una matriz $L \in \mathbb{R}^{n \times n}$ es \textbf{triangular inferior} si:
\[
L_{ij} = 0 \quad \text{para todo } i < j
\]

\textbf{Forma:}
\[
L = \begin{pmatrix}
l_{11} & 0 & 0 & \cdots & 0 \\
l_{21} & l_{22} & 0 & \cdots & 0 \\
\vdots & \vdots & \ddots & \ddots & \vdots \\
l_{n1} & l_{n2} & \cdots & l_{n,n-1} & l_{nn}
\end{pmatrix}
\]
\end{frame}

\begin{frame}{Matriz Triangular Superior}
\textbf{Definición:}
Una matriz $U \in \mathbb{R}^{n \times n}$ es \textbf{triangular superior} si:
\[
U_{ij} = 0 \quad \text{para todo } i > j
\]

\textbf{Forma:}
\[
U = \begin{pmatrix}
u_{11} & u_{12} & u_{13} & \cdots & u_{1n} \\
0 & u_{22} & u_{23} & \cdots & u_{2n} \\
\vdots & \ddots & \ddots & \ddots & \vdots \\
0 & \cdots & 0 & u_{n-1,n-1} & u_{n-1,n} \\
0 & \cdots & \cdots & 0 & u_{nn}
\end{pmatrix}
\]
\end{frame}

\begin{frame}{Descomposición LU}
\textbf{Definición:}
Una matriz $A \in \mathbb{R}^{n \times n}$ admite una \textbf{descomposición LU} si existen matrices:
\begin{itemize}
  \item $L$ triangular inferior con $L_{ii} = 1$ para todo $i$
  \item $U$ triangular superior
\end{itemize}
tales que:
\[
A = LU
\]
\end{frame}

\begin{frame}{Existencia y Unicidad}
\textbf{Teorema:}
Sea $A \in \mathbb{R}^{n \times n}$. Si todos los menores principales de $A$ son no singulares, entonces existe una única descomposición $A = LU$ con $L$ triangular inferior unitaria y $U$ triangular superior.

\vspace{0.3cm}
\textbf{Menor principal de orden $k$:}
\[
A_k = \begin{pmatrix}
a_{11} & \cdots & a_{1k} \\
\vdots & \ddots & \vdots \\
a_{k1} & \cdots & a_{kk}
\end{pmatrix}
\]
\end{frame}

\begin{frame}{Matriz de Permutación}
Cuando la condición de menores principales no se cumple, usamos \textbf{pivoteo parcial}.

\vspace{0.3cm}
\textbf{Definición:} Una matriz $P \in \mathbb{R}^{n \times n}$ es de permutación si:
\begin{itemize}
  \item Cada fila y columna tiene exactamente un 1
  \item Los demás elementos son 0
\end{itemize}

\vspace{0.3cm}
\textbf{Ejemplo:}
\[
P = \begin{pmatrix}
0 & 1 & 0 \\
1 & 0 & 0 \\
0 & 0 & 1
\end{pmatrix}
\]
intercambia las filas 1 y 2.
\end{frame}

\begin{frame}{Descomposición LU con Pivoteo Parcial}
\textbf{Teorema:}
Para toda matriz $A \in \mathbb{R}^{n \times n}$ no singular, existe una matriz de permutación $P$ tal que:
\[
PA = LU
\]
donde $L$ es triangular inferior unitaria y $U$ es triangular superior.

\vspace{0.3cm}
\textbf{Ventaja:} Mejora la estabilidad numérica al evitar divisiones por elementos muy pequeños.
\end{frame}

% ============================================================
% SECCIÓN 3: COMPLEJIDAD COMPUTACIONAL
% ============================================================
\section{Análisis de Complejidad Computacional}

\begin{frame}{Complejidad de la Factorización LU}
\textbf{Proposición:}
El algoritmo de factorización LU sin pivoteo requiere:
\[
\frac{2n^3}{3} + O(n^2) \text{ operaciones}
\]

\vspace{0.3cm}
\textbf{Justificación:}
Para cada columna $k$, se realizan:
\begin{itemize}
  \item Cálculo de multiplicadores: $n-k$ operaciones
  \item Actualización de la submatriz: $(n-k)^2$ operaciones
\end{itemize}

Total: $\displaystyle \sum_{k=1}^{n-1} (n-k)^2 = \frac{n(n-1)(2n-1)}{6} \approx \frac{n^3}{3}$
\end{frame}

\begin{frame}{Complejidad de la Resolución}
\textbf{Proposición:}
Resolver dos sistemas triangulares (sustitución progresiva y regresiva) requiere:
\[
2n^2 \text{ operaciones}
\]

\vspace{0.3cm}
\textbf{Desglose:}
\begin{itemize}
  \item Sustitución progresiva ($Ly = b$): $n^2$ operaciones
  \item Sustitución regresiva ($Ux = y$): $n^2$ operaciones
\end{itemize}
\end{frame}

\begin{frame}{Comparación LU vs Eliminación Gaussiana}
\textbf{Problema:} Resolver $m$ sistemas $Ax^{(k)} = b^{(k)}$ con la misma matriz $A$.

\vspace{0.3cm}
\begin{block}{Descomposición LU}
\[
T_{LU}(n,m) = \frac{2n^3}{3} + m \cdot 2n^2 + O(n^2)
\]
\end{block}

\begin{block}{Eliminación Gaussiana repetida}
\[
T_{Gauss}(n,m) = m \cdot \frac{2n^3}{3} + O(mn^2)
\]
\end{block}
\end{frame}

\begin{frame}{Speedup de la Descomposición LU}
\textbf{Factor de aceleración:}
\[
S(n,m) = \frac{T_{Gauss}}{T_{LU}} = \frac{m \cdot \frac{2n^3}{3}}{\frac{2n^3}{3} + m \cdot 2n^2} = \frac{m}{1 + \frac{3m}{n}}
\]

\vspace{0.3cm}
\textbf{Análisis:}
\begin{itemize}
  \item Si $m \ll n$: $S \approx m$ (speedup lineal en $m$)
  \item Si $m \gg n$: $S \approx \frac{n}{3}$ (speedup limitado por $n$)
  \item Para $n = 1000$, $m = 10$: $S \approx 9.7$ (cerca de 10x más rápido)
\end{itemize}
\end{frame}

% ============================================================
% SECCIÓN 4: MODELADO MATEMÁTICO
% ============================================================
\section{Modelado Matemático del Sistema Físico}

\begin{frame}{Descripción del Sistema}
Consideramos un sistema de $N$ partículas puntuales en $\mathbb{R}^2$:

\vspace{0.3cm}
\begin{itemize}
  \item \textbf{Posición:} $\vec{r}_i = (x_i, y_i)^T \in \mathbb{R}^2$
  \item \textbf{Masa:} $m_i > 0$
  \item \textbf{Conexiones:} Resortes elásticos entre pares de partículas
  \item \textbf{Frontera:} Algunas partículas están fijas en el espacio
\end{itemize}

\vspace{0.3cm}
\textbf{Objetivo:} Encontrar las posiciones de equilibrio de las partículas libres.
\end{frame}

\begin{frame}{Fuerzas en el Sistema: Gravedad}
\textbf{Fuerza gravitatoria:}
La gravedad actúa verticalmente hacia abajo:
\[
\vec{F}_i^{grav} = m_i \vec{g} = m_i \begin{pmatrix} 0 \\ -g \end{pmatrix}
\]
donde $g > 0$ es la aceleración gravitacional.
\end{frame}

\begin{frame}{Fuerzas en el Sistema: Resorte}
\textbf{Ley de Hooke:}
La fuerza ejercida por un resorte que conecta las partículas $i$ y $j$ es:
\[
\vec{F}_{ij} = -k_{ij}(\|\vec{r}_i - \vec{r}_j\| - L_{ij}^0) \frac{\vec{r}_i - \vec{r}_j}{\|\vec{r}_i - \vec{r}_j\|}
\]

\vspace{0.3cm}
donde:
\begin{itemize}
  \item $k_{ij}$: constante elástica del resorte
  \item $L_{ij}^0$: longitud natural del resorte
  \item $\|\vec{r}_i - \vec{r}_j\|$: longitud actual
\end{itemize}
\end{frame}

\begin{frame}{Condición de Equilibrio Estático}
En equilibrio, la suma de todas las fuerzas sobre cada partícula libre es cero:
\[
\vec{F}_i^{total} = \vec{F}_i^{grav} + \sum_{j \in \mathcal{N}(i)} \vec{F}_{ij} = \vec{0}
\]

donde $\mathcal{N}(i)$ es el conjunto de partículas conectadas a $i$ por resortes.

\vspace{0.3cm}
Esto genera un sistema de $2n$ ecuaciones no lineales (para $n$ partículas libres).
\end{frame}

\begin{frame}{Linealización de las Fuerzas}
\textbf{Hipótesis:} Pequeñas deformaciones respecto a la configuración de referencia.

\vspace{0.3cm}
Para un resorte entre $i$ y $j$ en equilibrio cerca de $L_{ij}^0$:
\[
\vec{F}_{ij} \approx -k_{ij}(\vec{r}_i - \vec{r}_j - \vec{L}_{ij}^0)
\]

donde $\vec{L}_{ij}^0$ es el vector de longitud natural.

\vspace{0.3cm}
\textbf{Resultado:} El sistema se convierte en \textbf{lineal}.
\end{frame}

% ============================================================
% SECCIÓN 5: FORMULACIÓN MATRICIAL
% ============================================================
\section{Formulación Matricial del Problema}

\begin{frame}{Matriz de Rigidez}
\textbf{Definición:}
Para un sistema con $n$ partículas libres, la matriz de rigidez $K \in \mathbb{R}^{2n \times 2n}$ se define por:
\[
K_{ij} = \begin{cases}
\displaystyle \sum_{m \in \mathcal{N}(i)} k_{im} & \text{si } i = j \\[10pt]
-k_{ij} & \text{si existe resorte entre } i \text{ y } j \\[10pt]
0 & \text{en otro caso}
\end{cases}
\]

\vspace{0.2cm}
\textbf{Nota:} En 2D, la matriz se construye en bloques $2 \times 2$ para cada par de partículas.
\end{frame}

\begin{frame}{Propiedades de la Matriz de Rigidez}
\textbf{Teorema:}
La matriz $K$ satisface:
\begin{enumerate}
  \item \textbf{Simetría:} $K = K^T$
  \item \textbf{Definida positiva:} Si hay al menos una partícula fija, $K$ es definida positiva
\end{enumerate}

\vspace{0.3cm}
\textbf{Implicación:} El sistema lineal $Kx = f$ tiene solución única.
\end{frame}

\begin{frame}{Vector de Fuerzas}
El vector de fuerzas externas $f \in \mathbb{R}^{2n}$ se construye como:
\[
f = \begin{pmatrix}
\vec{F}_1^{ext} \\
\vec{F}_2^{ext} \\
\vdots \\
\vec{F}_n^{ext}
\end{pmatrix}
\]

donde $\vec{F}_i^{ext}$ incluye:
\begin{itemize}
  \item Fuerzas gravitatorias: $m_i \vec{g}$
  \item Fuerzas de partículas fijas
  \item Otras fuerzas externas aplicadas
\end{itemize}
\end{frame}

\begin{frame}{Sistema Lineal Final}
\textbf{Ecuación matricial:}
\[
Kx = f
\]

donde:
\begin{itemize}
  \item $K \in \mathbb{R}^{2n \times 2n}$: matriz de rigidez (simétrica, definida positiva)
  \item $x \in \mathbb{R}^{2n}$: vector de posiciones desconocidas
  \item $f \in \mathbb{R}^{2n}$: vector de fuerzas externas
\end{itemize}

\vspace{0.3cm}
\textbf{Solución:} Usar descomposición LU para resolver eficientemente.
\end{frame}

% ============================================================
% SECCIÓN 6: IMPLEMENTACIÓN
% ============================================================
\section{Implementación Computacional}

\begin{frame}{Algoritmo de Descomposición LU}
\textbf{Entrada:} Matriz $A \in \mathbb{R}^{n \times n}$

\textbf{Salida:} Matrices $L$ y $U$ tales que $A = LU$

\vspace{0.3cm}
\textbf{Pasos principales:}
\begin{enumerate}
  \item Inicializar $L = I$, $U = A$
  \item Para $k = 1, \ldots, n-1$:
  \begin{itemize}
    \item Calcular multiplicadores: $L_{ik} = U_{ik}/U_{kk}$ para $i > k$
    \item Actualizar $U$: $U_{ij} = U_{ij} - L_{ik}U_{kj}$ para $i,j > k$
  \end{itemize}
\end{enumerate}

\vspace{0.3cm}
\textbf{Complejidad:} $O(n^3)$
\end{frame}

\begin{frame}{Algoritmo de Simulación}
\textbf{Proceso completo:}
\begin{enumerate}
  \item Definir configuración del sistema (partículas, resortes, masas)
  \item Construir matriz de rigidez $K$
  \item Calcular vector de fuerzas externas $f$
  \item Factorizar: $K = LU$ (una sola vez)
  \item Para diferentes configuraciones de fuerzas:
  \begin{itemize}
    \item Resolver $Ly = f$ (sustitución progresiva)
    \item Resolver $Ux = y$ (sustitución regresiva)
  \end{itemize}
  \item Visualizar resultados
\end{enumerate}
\end{frame}

\begin{frame}{Estructura del Código}
\textbf{Bibliotecas utilizadas:}
\begin{itemize}
  \item \texttt{numpy}: Operaciones matriciales
  \item \texttt{scipy.linalg}: Descomposición LU optimizada
  \item \texttt{matplotlib}: Visualización
\end{itemize}

\vspace{0.3cm}
\textbf{Módulos principales:}
\begin{itemize}
  \item Construcción de la matriz de rigidez
  \item Aplicación de condiciones de frontera
  \item Resolución mediante LU
  \item Graficación del sistema de partículas
\end{itemize}
\end{frame}

% ============================================================
% SECCIÓN 7: RESULTADOS
% ============================================================
\section{Resultados Experimentales}

\begin{frame}{Configuración de Pruebas}
\textbf{Sistema de prueba:}
\begin{itemize}
  \item 20 partículas en configuración de malla 2D
  \item 4 partículas fijas en las esquinas
  \item Constantes elásticas uniformes: $k = 10$ N/m
  \item Masa uniforme: $m = 1$ kg
  \item Gravedad: $g = 9.8$ m/s²
\end{itemize}

\vspace{0.3cm}
\textbf{Mediciones:}
\begin{itemize}
  \item Tiempo de factorización
  \item Tiempo de resolución
  \item Precisión de la solución
\end{itemize}
\end{frame}

\begin{frame}{Análisis de Resultados}
\textbf{Observaciones principales:}
\begin{itemize}
  \item La factorización LU es significativamente más rápida que Gauss para múltiples sistemas
  \item El speedup observado coincide con la predicción teórica
  \item La solución converge correctamente a la configuración de equilibrio
  \item Errores numéricos en el orden de $10^{-12}$ (precisión de máquina)
\end{itemize}

\vspace{0.3cm}
\textbf{Ventaja computacional:}
Para $m = 10$ sistemas con $n = 100$ variables:
\begin{itemize}
  \item LU: 0.15 segundos
  \item Gauss: 1.42 segundos
  \item Speedup: $9.5\times$
\end{itemize}
\end{frame}

% ============================================================
% CONCLUSIONES
% ============================================================
\section{Conclusiones}

\begin{frame}{Conclusiones}
\begin{enumerate}
  \item La descomposición LU es \textbf{altamente eficiente} para resolver múltiples sistemas lineales con la misma matriz.
  
  \item El speedup teórico se \textbf{confirma experimentalmente} en la aplicación práctica.
  
  \item El modelado de sistemas físicos mediante matrices de rigidez genera sistemas lineales bien condicionados.
  
  \item La combinación de fundamentos matemáticos rigurosos con implementación computacional eficiente es esencial para aplicaciones en ingeniería.
\end{enumerate}
\end{frame}

\begin{frame}{Trabajo Futuro}
\textbf{Extensiones posibles:}
\begin{itemize}
  \item Implementación de descomposición de Cholesky (aprovechando simetría)
  \item Estudio de sistemas dinámicos (ecuaciones de movimiento)
  \item Análisis de estabilidad numérica con matrices mal condicionadas
  \item Paralelización del algoritmo para sistemas muy grandes
  \item Aplicación a problemas en 3D
\end{itemize}
\end{frame}

\begin{frame}[allowframebreaks]{Referencias}
\begin{thebibliography}{99}
\bibitem{golub2013matrix} G. H. Golub, C. F. Van Loan. \textit{Matrix Computations}. Johns Hopkins University Press, 2013.

\bibitem{strang2016introduction} G. Strang. \textit{Introduction to Linear Algebra}. Wellesley-Cambridge Press, 2016.

\bibitem{burden2015numerical} R. L. Burden, J. D. Faires. \textit{Numerical Analysis}. Cengage Learning, 2015.

\bibitem{hibbeler2016engineering} R. C. Hibbeler. \textit{Engineering Mechanics: Statics}. Pearson, 2016.

\bibitem{cook2007concepts} R. D. Cook. \textit{Concepts and Applications of Finite Element Analysis}. Wiley, 2007.

\bibitem{zienkiewicz2013finite} O. C. Zienkiewicz, R. L. Taylor. \textit{The Finite Element Method}. Butterworth-Heinemann, 2013.
\end{thebibliography}
\end{frame}

\begin{frame}
\centering
\Huge ¡Gracias por su atención!

\vspace{1cm}
\Large ¿Preguntas?
\end{frame}

\end{document}
