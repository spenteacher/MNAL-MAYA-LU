\documentclass[10pt]{beamer}
\usepackage[utf8]{inputenc}
\usepackage[spanish]{babel}
\usepackage{amsmath, amssymb, amsthm}
\usepackage{hyperref}
\usepackage{graphicx}

\usetheme{JuanLesPins}
\usecolortheme{default}

\title{Descomposición LU en Simulación Computacional}
\subtitle{Fundamentos Teóricos y Aplicación a Sistemas de Partículas}
\author{S. Zaragoza Pastor \and J. Hernández Barrero \and B. Liu  \newline \and A. Fernández García \and P. Botella Guillén}
\date{}

\begin{document}

\begin{frame}
  \titlepage
\end{frame}

\begin{frame}{Contenido}
  \tableofcontents
\end{frame}

% ============================================================
% SECCIÓN 1: INTRODUCCIÓN
% ============================================================
\section{Introducción}

\begin{frame}{Motivación}
La resolución de sistemas lineales $Ax = b$ es fundamental en álgebra lineal computacional.

\vspace{0.3cm}
\textbf{Pregunta clave:} ¿Podemos aprovechar cálculos previos al resolver múltiples sistemas con la misma matriz $A$ pero diferentes vectores $b$?

\vspace{0.3cm}
\textbf{Respuesta:} La descomposición LU separa el problema en dos fases:
\begin{enumerate}
  \item \textbf{Factorización:} $A = LU$ (una sola vez)
  \item \textbf{Resolución:} Resolver $Ly = b$ y $Ux = y$ (para cada $b$)
\end{enumerate}
\end{frame}

\begin{frame}{Aplicación Práctica}
\textbf{Sistema de partículas conectadas por resortes:}
\begin{itemize}
  \item Modelado mediante ecuaciones de equilibrio estático
  \item Resulta en sistemas lineales grandes $Kx = f$
  \item La matriz $K$ (rigidez) es simétrica y definida positiva
  \item Necesidad de resolver múltiples configuraciones eficientemente
\end{itemize}
\end{frame}

% ============================================================
% SECCIÓN 2: FUNDAMENTOS TEÓRICOS
% ============================================================
\section{Fundamentos Teóricos de la Descomposición LU}

\begin{frame}{Matriz Triangular Inferior}
\textbf{Definición:}
Una matriz $L \in \mathbb{R}^{n \times n}$ es \textbf{triangular inferior} si:
\[
L_{ij} = 0 \quad \text{para todo } i < j
\]

\textbf{Forma:}
\[
L = \begin{pmatrix}
l_{11} & 0 & 0 & \cdots & 0 \\
l_{21} & l_{22} & 0 & \cdots & 0 \\
\vdots & \vdots & \ddots & \ddots & \vdots \\
l_{n1} & l_{n2} & \cdots & l_{n,n-1} & l_{nn}
\end{pmatrix}
\]
\end{frame}

\begin{frame}{Matriz Triangular Superior}
\textbf{Definición:}
Una matriz $U \in \mathbb{R}^{n \times n}$ es \textbf{triangular superior} si:
\[
U_{ij} = 0 \quad \text{para todo } i > j
\]

\textbf{Forma:}
\[
U = \begin{pmatrix}
u_{11} & u_{12} & u_{13} & \cdots & u_{1n} \\
0 & u_{22} & u_{23} & \cdots & u_{2n} \\
\vdots & \ddots & \ddots & \ddots & \vdots \\
0 & \cdots & 0 & u_{n-1,n-1} & u_{n-1,n} \\
0 & \cdots & \cdots & 0 & u_{nn}
\end{pmatrix}
\]
\end{frame}

\begin{frame}{Descomposición LU}
\textbf{Definición:}
Una matriz $A \in \mathbb{R}^{n \times n}$ admite una \textbf{descomposición LU} si existen matrices:
\begin{itemize}
  \item $L$ triangular inferior con $L_{ii} = 1$ para todo $i$
  \item $U$ triangular superior
\end{itemize}
tales que:
\[
A = LU
\]
\end{frame}


\begin{frame}{Matriz de Permutación}
Para evitar dividir por cero en alguna iteración, usamos \textbf{pivoteo parcial}.

\vspace{0.3cm}
\textbf{Definición:} Una matriz $P \in \mathbb{R}^{n \times n}$ es de permutación si:
\begin{itemize}
  \item Cada fila y columna tiene exactamente un 1
  \item Los demás elementos son 0
\end{itemize}

\vspace{0.3cm}
\textbf{Ejemplo:}
\[
P = \begin{pmatrix}
0 & 1 & 0 \\
1 & 0 & 0 \\
0 & 0 & 1
\end{pmatrix}
\]
intercambia las filas 1 y 2.
\end{frame}

\begin{frame}{Descomposición LU con Pivoteo Parcial}
\textbf{Teorema:}
Para toda matriz $A \in \mathbb{R}^{n \times n}$ no singular, existe una matriz de permutación $P$ tal que:
\[
PA = LU
\]
donde $L$ es triangular inferior unitaria y $U$ es triangular superior.

\vspace{0.3cm}
\textbf{Ventaja:} Mejora la estabilidad numérica al evitar divisiones por elementos muy pequeños.

\vspace{0.3cm}
\textbf{Corolario:} Para toda matriz definida positiva, se puede llevar a cabo la descomposición LU con pivoteo parcial.

\end{frame}

% ============================================================
% SECCIÓN 3: COMPLEJIDAD COMPUTACIONAL
% ============================================================
\section{Análisis de Complejidad Computacional}

\begin{frame}{Complejidad de la Factorización LU}
\textbf{Proposición:}
El algoritmo de factorización LU sin pivoteo requiere:
\[
\frac{2n^3}{3} + O(n^2) \text{ operaciones}
\]

\vspace{0.3cm}
\textbf{Justificación:}
Para cada columna $k$, se realizan:
\begin{itemize}
  \item Cálculo de multiplicadores: $n-k$ operaciones
  \item Actualización de la submatriz: $(n-k)^2$ operaciones
\end{itemize}

Total: $\displaystyle \sum_{k=1}^{n-1} (n-k)^2 = \frac{n(n-1)(2n-1)}{6} \approx \frac{n^3}{3}$
\end{frame}

\begin{frame}{Complejidad de la Resolución}
\textbf{Proposición:}
Resolver dos sistemas triangulares (sustitución progresiva y regresiva) requiere:
\[
2n^2 \text{ operaciones}
\]

\vspace{0.3cm}
\textbf{Desglose:}
\begin{itemize}
  \item Sustitución progresiva ($Ly = b$): $n^2$ operaciones
  \item Sustitución regresiva ($Ux = y$): $n^2$ operaciones
\end{itemize}
\end{frame}

\begin{frame}{Comparación LU vs Eliminación Gaussiana}
\textbf{Problema:} Resolver $m$ sistemas $Ax = b$ con la misma matriz $A$.

\vspace{0.3cm}
\begin{block}{Descomposición LU}
\[
T_{LU}(n,m) = \frac{2n^3}{3} - \frac{n^2}{2} - \frac{n}{6} +  m \cdot 2n^2 \approx  \frac{2n^3}{3} + m \cdot 2n^2 + O(n^2)
\]
\end{block}

\begin{block}{Eliminación Gaussiana repetida}
\[
T_{Gauss}(n,m) = m(\frac{2n^3}{3} - \frac{n^2}{2} - \frac{n}{6} +n^2) \approx m \cdot \frac{2n^3}{3} + O(mn^2)
\]
\end{block}

\textbf{Factor de aceleración:}
\[
S(n,m) = \frac{T_{Gauss}}{T_{LU}} = \frac{m(\frac{2n^3}{3} - \frac{n^2}{2} - \frac{n}{6} +n^2)}{\frac{2n^3}{3} - \frac{n^2}{2} - \frac{n}{6} +  m \cdot 2n^2} \approx \frac{m}{1 + \frac{3m}{n}}
\]

\end{frame}


% ============================================================
% SECCIÓN 4: MODELADO MATEMÁTICO
% ============================================================
\section{Modelado Matemático del Sistema Físico}

\begin{frame}{Descripción del Sistema}
Consideramos un sistema de $N$ partículas puntuales en $\mathbb{R}^2$:

\vspace{0.3cm}
\begin{itemize}
  \item \textbf{Posición de la partícula}  \(i\): \(\vec{r}_i = (x_i, y_i)^T \in \mathbb{R}^2\)
  \item \textbf{Masa:} $m_i > 0$
  \item \textbf{Conexiones:} Resortes elásticos entre pares de partículas
  \item \textbf{Frontera:} Algunas partículas están fijas en el espacio
\end{itemize}

\vspace{0.3cm}
\textbf{Objetivo:} Encontrar las posiciones de equilibrio de las partículas libres.
\end{frame}

\begin{frame}{Fuerzas en el Sistema: Gravedad}
\textbf{Fuerza gravitatoria:}
La gravedad actúa verticalmente hacia abajo:
\[
\vec{F}_i^{grav} = m_i \vec{g} = m_i \begin{pmatrix} 0 \\ -g \end{pmatrix}
\]
donde $g > 0$ es la aceleración gravitacional.
\end{frame}

\begin{frame}{Fuerzas en el Sistema: Resorte}
\textbf{Ley de Hooke:}
La fuerza ejercida por un resorte que conecta las partículas $i$ y $j$ es:
\[
\vec{F}_{ij} = -k_{ij}(\|\vec{r}_i - \vec{r}_j\| - L_{ij}^0) \frac{\vec{r}_i - \vec{r}_j}{\|\vec{r}_i - \vec{r}_j\|}
\]

\vspace{0.3cm}
donde:
\begin{itemize}
  \item $k_{ij} > 0$: constante elástica del resorte
  \item $L_{ij}^0 > 0$: longitud natural del resorte
  \item $\|\vec{r}_i - \vec{r}_j\|$: longitud actual
  \item \(\hat{e}_{ij} = \frac{\vec{r}_j - \vec{r}_i}{|\vec{r}_j - \vec{r}_i|}\): Vector unitario de \(i\) hacia \(j\)
\end{itemize}

\begin{block}{Nota}
	La fuerza \(\vec{F}_{ij}\) actúa sobre la partícula \(i\) en dirección \(i \to j\). Por la tercera ley de Newton:
	\[
	\vec{F}_{ji} = -\vec{F}_{ij}
	\]
\end{block}

\end{frame}

\begin{frame}{Condición de Equilibrio Estático}
	
\begin{block}{Equilibrio Estático}
	Un sistema de partículas está en \textbf{equilibrio estático} si la fuerza total sobre cada partícula libre es nula:
	\[
	\vec{F}_i^{grav} + \sum_{j \in \mathcal{N}(i)} \vec{F}_{ij} = \vec{0} \quad \forall i \text{ libre}
	\]
	donde \(\mathcal{N}(i)\) es el conjunto de partículas conectadas a \(i\).
\end{block}


\vspace{0.3cm}
\end{frame}

\begin{frame}{Linealización de las Fuerzas}
\textbf{Hipótesis:} Pequeñas deformaciones respecto a la configuración de referencia.

\vspace{0.3cm}
Para un resorte entre $i$ y $j$ en equilibrio cerca de $L_{ij}^0$:
\[
\vec{F}_{ij} \approx -k_{ij}(\vec{r}_i - \vec{r}_j - \vec{L}_{ij}^0)
\]

donde $\vec{L}_{ij}^0$ es el vector de longitud natural.

\vspace{0.3cm}
\textbf{Resultado:} El sistema se convierte en \textbf{lineal}.
\end{frame}

\begin{frame}{Proof}
	Sea \(\vec{\delta}_i = \vec{r}_i - \vec{r}_i^0\) el desplazamiento desde la posición de reposo. Entonces:
	\[
	|\vec{r}_j - \vec{r}_i| = |(\vec{r}_j^0 - \vec{r}_i^0) + (\vec{\delta}_j - \vec{\delta}_i)|
	\]
	
	Para \(|\vec{\delta}_j - \vec{\delta}_i| \ll L_{ij}^0\), expandimos:
	\[
	|\vec{r}_j - \vec{r}_i| \approx L_{ij}^0 + \frac{(\vec{r}_j^0 - \vec{r}_i^0) \cdot (\vec{\delta}_j - \vec{\delta}_i)}{L_{ij}^0}
	\]
	
	Sustituyendo en la ley de Hooke y simplificando, obtenemos la aproximación lineal.
\end{frame}

% ============================================================
% SECCIÓN 5: FORMULACIÓN MATRICIAL
% ============================================================
\section{Formulación Matricial del Problema}

\begin{frame}{Matriz de Rigidez}
\textbf{Definición:}
Para un sistema con $n$ partículas libres, la matriz de rigidez $K \in \mathbb{R}^{2n \times 2n}$ se define por:
\[
K_{ij} = \begin{cases}
\displaystyle \sum_{m \in \mathcal{N}(i)} k_{im} & \text{si } i = j \\[10pt]
-k_{ij} & \text{si existe resorte entre } i \text{ y } j \\[10pt]
0 & \text{en otro caso}
\end{cases}
\]
\vspace{0.2cm}

\begin{block}{Propiedades}
	\begin{enumerate}
		\item \textbf{Simetría}: \(K = K^T\)
		\item \textbf{Definida positiva}: Si hay al menos una partícula fija, \(K\) es definida positiva
	\end{enumerate}
\end{block}

	\textbf{\textit{Nota: }}En nuestro proyecto y y programa en la práctica se aproxima la matriz de Rigidez a una matriz de un sistema unidimensional, luego podemos asociar una componente en función de la otra.
\end{frame}


\begin{frame}{Proof}
	\textbf{(1) Simetría}: Por construcción, si existe resorte entre \(i\) y \(j\), entonces \(K_{ij} = K_{ji} = -k_{ij}\).
	
	\textbf{(2) Definida positividad}: Consideremos la energía potencial elástica total:
	\[
	V(\vec{x}) = \frac{1}{2} \sum_{\text{resortes } ij} k_{ij} (|\vec{r}_j - \vec{r}_i| - L_{ij}^0)^2
	\]
	
	En la aproximación lineal:
	\[
	V \approx \frac{1}{2} \vec{x}^T K \vec{x} + \text{constante}
	\]
	
	Como \(V \geq 0\) y alcanza el mínimo solo en el equilibrio, \(K\) debe ser semidefinida positiva. La presencia de partículas fijas elimina los modos rígidos (traslaciones), haciendo \(K\) definida positiva.
\end{frame}


\begin{frame}{Vector de Fuerzas}
\begin{definition}[Vector de Términos Independientes]
	El vector \(\vec{f} \in \mathbb{R}^n\) se construye como:
	\[
	\vec{f} = K \vec{x}^0 + \vec{F}^{ext}
	\]
	donde:
	\begin{itemize}
		\item \(\vec{x}^0\): Vector de posiciones de reposo
		\item \(\vec{F}^{ext}\): Fuerzas externas (gravedad)
	\end{itemize}
\end{definition}

\end{frame}

\begin{frame}{Sistema Lineal Final}
\begin{theorem}[Formulación del Equilibrio como Sistema Lineal]
	Las posiciones de equilibrio \(\vec{x}_{eq}\) del sistema satisfacen:
	\[
	K \vec{x}_{eq} = \vec{F}^{ext}
	\]
	Este sistema se resuelve independientemente para cada coordenada espacial.
\end{theorem}
\end{frame}


% ============================================================
% SECCIÓN 6: IMPLEMENTACIÓN
% ============================================================
\section{Implementación Computacional}

\begin{frame}{Estructura del Código}
	\textbf{Módulos:}
	\begin{itemize}
		\item \texttt{mainsimple.py}: Interfaz gráfica y preparación del sistema
		\item \texttt{physicssimple.py}: Aplicación de las fuerzas al sistema de partículas
		\item \texttt{lusolver.py}: Algoritmo LU y Gauss clásico
	\end{itemize}
\end{frame}

\begin{frame}
\centering
\Huge ¡Gracias por su atención!

\vspace{1cm}
\Large ¿Preguntas?
\end{frame}

\end{document}
